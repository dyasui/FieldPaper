% Options for packages loaded elsewhere
\PassOptionsToPackage{unicode}{hyperref}
\PassOptionsToPackage{hyphens}{url}
\PassOptionsToPackage{dvipsnames,svgnames,x11names}{xcolor}
%
\documentclass[
]{article}

\usepackage{amsmath,amssymb}
\usepackage{iftex}
\ifPDFTeX
  \usepackage[T1]{fontenc}
  \usepackage[utf8]{inputenc}
  \usepackage{textcomp} % provide euro and other symbols
\else % if luatex or xetex
  \usepackage{unicode-math}
  \defaultfontfeatures{Scale=MatchLowercase}
  \defaultfontfeatures[\rmfamily]{Ligatures=TeX,Scale=1}
\fi
\usepackage{lmodern}
\ifPDFTeX\else  
    % xetex/luatex font selection
\fi
% Use upquote if available, for straight quotes in verbatim environments
\IfFileExists{upquote.sty}{\usepackage{upquote}}{}
\IfFileExists{microtype.sty}{% use microtype if available
  \usepackage[]{microtype}
  \UseMicrotypeSet[protrusion]{basicmath} % disable protrusion for tt fonts
}{}
\makeatletter
\@ifundefined{KOMAClassName}{% if non-KOMA class
  \IfFileExists{parskip.sty}{%
    \usepackage{parskip}
  }{% else
    \setlength{\parindent}{0pt}
    \setlength{\parskip}{6pt plus 2pt minus 1pt}}
}{% if KOMA class
  \KOMAoptions{parskip=half}}
\makeatother
\usepackage{xcolor}
\usepackage[left=1in,right=1in]{geometry}
\setlength{\emergencystretch}{3em} % prevent overfull lines
\setcounter{secnumdepth}{-\maxdimen} % remove section numbering
% Make \paragraph and \subparagraph free-standing
\makeatletter
\ifx\paragraph\undefined\else
  \let\oldparagraph\paragraph
  \renewcommand{\paragraph}{
    \@ifstar
      \xxxParagraphStar
      \xxxParagraphNoStar
  }
  \newcommand{\xxxParagraphStar}[1]{\oldparagraph*{#1}\mbox{}}
  \newcommand{\xxxParagraphNoStar}[1]{\oldparagraph{#1}\mbox{}}
\fi
\ifx\subparagraph\undefined\else
  \let\oldsubparagraph\subparagraph
  \renewcommand{\subparagraph}{
    \@ifstar
      \xxxSubParagraphStar
      \xxxSubParagraphNoStar
  }
  \newcommand{\xxxSubParagraphStar}[1]{\oldsubparagraph*{#1}\mbox{}}
  \newcommand{\xxxSubParagraphNoStar}[1]{\oldsubparagraph{#1}\mbox{}}
\fi
\makeatother


\providecommand{\tightlist}{%
  \setlength{\itemsep}{0pt}\setlength{\parskip}{0pt}}\usepackage{longtable,booktabs,array}
\usepackage{calc} % for calculating minipage widths
% Correct order of tables after \paragraph or \subparagraph
\usepackage{etoolbox}
\makeatletter
\patchcmd\longtable{\par}{\if@noskipsec\mbox{}\fi\par}{}{}
\makeatother
% Allow footnotes in longtable head/foot
\IfFileExists{footnotehyper.sty}{\usepackage{footnotehyper}}{\usepackage{footnote}}
\makesavenoteenv{longtable}
\usepackage{graphicx}
\makeatletter
\def\maxwidth{\ifdim\Gin@nat@width>\linewidth\linewidth\else\Gin@nat@width\fi}
\def\maxheight{\ifdim\Gin@nat@height>\textheight\textheight\else\Gin@nat@height\fi}
\makeatother
% Scale images if necessary, so that they will not overflow the page
% margins by default, and it is still possible to overwrite the defaults
% using explicit options in \includegraphics[width, height, ...]{}
\setkeys{Gin}{width=\maxwidth,height=\maxheight,keepaspectratio}
% Set default figure placement to htbp
\makeatletter
\def\fps@figure{htbp}
\makeatother
% definitions for citeproc citations
\NewDocumentCommand\citeproctext{}{}
\NewDocumentCommand\citeproc{mm}{%
  \begingroup\def\citeproctext{#2}\cite{#1}\endgroup}
\makeatletter
 % allow citations to break across lines
 \let\@cite@ofmt\@firstofone
 % avoid brackets around text for \cite:
 \def\@biblabel#1{}
 \def\@cite#1#2{{#1\if@tempswa , #2\fi}}
\makeatother
\newlength{\cslhangindent}
\setlength{\cslhangindent}{1.5em}
\newlength{\csllabelwidth}
\setlength{\csllabelwidth}{3em}
\newenvironment{CSLReferences}[2] % #1 hanging-indent, #2 entry-spacing
 {\begin{list}{}{%
  \setlength{\itemindent}{0pt}
  \setlength{\leftmargin}{0pt}
  \setlength{\parsep}{0pt}
  % turn on hanging indent if param 1 is 1
  \ifodd #1
   \setlength{\leftmargin}{\cslhangindent}
   \setlength{\itemindent}{-1\cslhangindent}
  \fi
  % set entry spacing
  \setlength{\itemsep}{#2\baselineskip}}}
 {\end{list}}
\usepackage{calc}
\newcommand{\CSLBlock}[1]{\hfill\break\parbox[t]{\linewidth}{\strut\ignorespaces#1\strut}}
\newcommand{\CSLLeftMargin}[1]{\parbox[t]{\csllabelwidth}{\strut#1\strut}}
\newcommand{\CSLRightInline}[1]{\parbox[t]{\linewidth - \csllabelwidth}{\strut#1\strut}}
\newcommand{\CSLIndent}[1]{\hspace{\cslhangindent}#1}

\makeatletter
\@ifpackageloaded{caption}{}{\usepackage{caption}}
\AtBeginDocument{%
\ifdefined\contentsname
  \renewcommand*\contentsname{Table of contents}
\else
  \newcommand\contentsname{Table of contents}
\fi
\ifdefined\listfigurename
  \renewcommand*\listfigurename{List of Figures}
\else
  \newcommand\listfigurename{List of Figures}
\fi
\ifdefined\listtablename
  \renewcommand*\listtablename{List of Tables}
\else
  \newcommand\listtablename{List of Tables}
\fi
\ifdefined\figurename
  \renewcommand*\figurename{Figure}
\else
  \newcommand\figurename{Figure}
\fi
\ifdefined\tablename
  \renewcommand*\tablename{Table}
\else
  \newcommand\tablename{Table}
\fi
}
\@ifpackageloaded{float}{}{\usepackage{float}}
\floatstyle{ruled}
\@ifundefined{c@chapter}{\newfloat{codelisting}{h}{lop}}{\newfloat{codelisting}{h}{lop}[chapter]}
\floatname{codelisting}{Listing}
\newcommand*\listoflistings{\listof{codelisting}{List of Listings}}
\makeatother
\makeatletter
\makeatother
\makeatletter
\@ifpackageloaded{caption}{}{\usepackage{caption}}
\@ifpackageloaded{subcaption}{}{\usepackage{subcaption}}
\makeatother
\ifLuaTeX
  \usepackage{selnolig}  % disable illegal ligatures
\fi
\usepackage{bookmark}

\IfFileExists{xurl.sty}{\usepackage{xurl}}{} % add URL line breaks if available
\urlstyle{same} % disable monospaced font for URLs
\hypersetup{
  pdftitle={Ethnic Enclaves and the Legacy of Internment},
  colorlinks=true,
  linkcolor={blue},
  filecolor={Maroon},
  citecolor={Blue},
  urlcolor={Blue},
  pdfcreator={LaTeX via pandoc}}

\title{Ethnic Enclaves and the Legacy of Internment}
\author{Dante Yasui}
\date{}

\begin{document}
\maketitle

\subsection{Population and Migration}\label{population-and-migration}

For long-run migration data, I use the Decennial Census data provided by
the US Census Bureau via the
\href{https://usa.ipums.org/usa/index.shtml}{Integrated Public Use
Microdata Series} (\citeproc{ref-ruggles_steven_ipums_2020}{Ruggles,
Steven et al. 2020}). The census samples for which county locations are
available include the 1940, 1950, 1980, and 1990 1\% samples, the 1960
5\% sample, and the 1970 Form 1 Metro 1\% sample.

For the calculation of migration rates between counties, I define a
migrant as someone who reports that they either moved within the state,
between states, or that they were abroad five years ago (or in the past
1 year for 1960 respondants). This excludes people who report moving
within the same house, didn't report their previous location, or the
location is unknown.

\subsection{Geography}\label{geography}

Locations for historical interment camp locations were archived by
\href{http://encyclopedia.densho.org/War_Relocation_Authority/\#Planning_the_Camps}{Densho
Encyclopedia} and downloaded in csv form via the
\href{https://www.arcgis.com/home/item.html?id=69183af8d45d4f46a9dc4eba99440891}{Behind
Barbed Wires story project}.

Although most county borders did not change much in the second half of
the 20th Century, there were counties which split, merged, or had name
changes which can make cross-decade comparisons difficult. For these
reasons, I choose to standardize the set of counties in my analysis to
the set of counties as they appear in the year 1990. To map historical
county-level data to 1990 county definitions, I implement the crosswalk
method by (\citeproc{ref-eckert_method_2020}{Eckert et al. 2020}). They
overlay historical county boundary shapefiles from
\href{NHGIS}{https://www.nhgis.org/} onto county boundaries for a
specific target year (in this case 1990). The sub-areas created by these
overlays are used to calculate a set of geographic weights which
represent the fractions of a 1990 county's area which were within the
geographic areas of counties as they appear in different decades
(specifically the decades between and including 1940 to 1980). For my
analysis, I take the crosswalk weights from the example csv file for the
end year 1990 which is published on the authors'
\href{https://github.com/liang-jack-a/EGLP_Crosswalk/tree/master}{github
repository}.

I calculate the straight line distances in meters between each 1950
county centroid to each camp location in QGIS with the
\texttt{Distance\ Matrix} tool using the Standard (N x T) distance
matrix setting.

\subsection{County-year level summary
statistics}\label{county-year-level-summary-statistics}

After narrowing down to counties which can be observed in each census
year and then translating the historical counties to 1990 county
boundaries, I am left with 3082 counties with observable migration rates
in 1940, 108 in 1950, 410 in 1960, 115 in 1970, 254 in 1980, and 290 in
1990.

\phantomsection\label{refs}
\begin{CSLReferences}{1}{0}
\bibitem[\citeproctext]{ref-eckert_method_2020}
Eckert, Fabian, Andrés Gvirtz, Jack Liang, and Michael Peters. 2020.
{``A {Method} to {Construct} {Geographical} {Crosswalks} with an
{Application} to {US} {Counties} Since 1790.''} Working \{Paper\}.
Working {Paper} {Series}. National Bureau of Economic Research.
\url{https://doi.org/10.3386/w26770}.

\bibitem[\citeproctext]{ref-ruggles_steven_ipums_2020}
Ruggles, Steven, Flood, Sarah, Goeken, Ronald, Schouweiler, Megan, and
Sobek, Matthew. 2020. {``{IPUMS} {USA}: {Version} 12.0 {[}Dataset{]}.''}
\url{https://usa.ipums.org/usa/cite.shtml}.

\end{CSLReferences}



\end{document}
