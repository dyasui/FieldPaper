
% Table created by stargazer v.5.2.3 by Marek Hlavac, Social Policy Institute. E-mail: marek.hlavac at gmail.com
% Date and time: Wed, Sep 11, 2024 - 13:50:20
\begin{table}[!htbp] \centering 
  \caption{} 
  \label{} 
\begin{tabular}{@{\extracolsep{5pt}}lcccccc} 
\\[-1.8ex]\hline 
\hline \\[-1.8ex] 
 & \multicolumn{6}{c}{\textit{Dependent variable:}} \\ 
\cline{2-7} 
\\[-1.8ex] & \multicolumn{6}{c}{y} \\ 
\\[-1.8ex] & (1) & (2) & (3) & (4) & (5) & (6)\\ 
\hline \\[-1.8ex] 
 campclosest\_dist & $-$0.000$^{***}$ & $-$0.000 & $-$0.000$^{***}$ & $-$0.000$^{***}$ & $-$0.000 & $-$0.000 \\ 
  & (0.000) & (0.000) & (0.000) & (0.000) & (0.000) & (0.000) \\ 
  & & & & & & \\ 
 ez & 0.002 & $-$0.001 & 0.002$^{**}$ & $-$0.002 & 0.002 & 0.001 \\ 
  & (0.001) & (0.003) & (0.001) & (0.001) & (0.002) & (0.001) \\ 
  & & & & & & \\ 
 campclosest\_dist:ez & 0.000$^{**}$ & 0.000 & 0.000 & 0.000$^{***}$ & 0.000$^{*}$ & 0.000$^{***}$ \\ 
  & (0.000) & (0.000) & (0.000) & (0.000) & (0.000) & (0.000) \\ 
  & & & & & & \\ 
 Constant & 0.001$^{***}$ & 0.001 & 0.002$^{***}$ & 0.004$^{***}$ & 0.001$^{*}$ & 0.002$^{***}$ \\ 
  & (0.0002) & (0.001) & (0.0003) & (0.001) & (0.001) & (0.001) \\ 
  & & & & & & \\ 
\hline \\[-1.8ex] 
Observations & 3,056 & 108 & 407 & 115 & 252 & 289 \\ 
R$^{2}$ & 0.045 & 0.040 & 0.272 & 0.560 & 0.280 & 0.292 \\ 
Adjusted R$^{2}$ & 0.044 & 0.012 & 0.267 & 0.548 & 0.272 & 0.284 \\ 
Residual Std. Error & 0.005 (df = 3052) & 0.003 (df = 104) & 0.002 (df = 403) & 0.002 (df = 111) & 0.003 (df = 248) & 0.003 (df = 285) \\ 
F Statistic & 47.907$^{***}$ (df = 3; 3052) & 1.427 (df = 3; 104) & 50.239$^{***}$ (df = 3; 403) & 47.128$^{***}$ (df = 3; 111) & 32.185$^{***}$ (df = 3; 248) & 39.131$^{***}$ (df = 3; 285) \\ 
\hline 
\hline \\[-1.8ex] 
\textit{Note:}  & \multicolumn{6}{r}{$^{*}$p$<$0.1; $^{**}$p$<$0.05; $^{***}$p$<$0.01} \\ 
\end{tabular} 
\end{table} 
